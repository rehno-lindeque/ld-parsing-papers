
\documentclass[11pt]{article}

%%%%%%%%%%%%%%%%%%% PACKAGES %%%%%%%%%%%%%%%%%%%
\usepackage{geometry}
\geometry{a4paper}
\usepackage{gcl}
%\usepackage{pfs}
%\usepackage{algorithmic}
%\usepackage{graphicx}
%\usepackage{epstopdf}
%\usepackage{natbib}
%\usepackage{fancyvrb}
\usepackage{float}
\usepackage{wrapfig}

\usepackage{amssymb}
\usepackage{amsthm}
\usepackage{amsmath}
\usepackage{proof}

%%%%%%%%%%%%%%%%%%%% MACROS %%%%%%%%%%%%%%%%%%%%
% Grammars
%\newcommand {\GRAMMAR}
%  {\begin{indent} \begin{align*}}
%  %{\setlength{\parindent}{\GRAMMAR}\\
%  %\begin{indent}}

%\newcommand {\ENDGRAMMAR}
%  {\end{align*} \end{indent}}

\newenvironment{grammar}{\begin{align*}}{\end{align*}}
  
%  {\begin{indent} \begin{align*}}
%  {\end{align*} \end{indent}}
  
%\setlength{\topsep}{2mm}<
%\setlength{\parindent}{15pt}

%\newlength{\OBLS}\setlength{\OBLS}{\baselineskip}
%\newlength{\OPAR}\setlength{\OPAR}{\parindent}
%\newcommand{\pgskip}{\hspace{\mathindent}\=\+}
%\newcommand{\FOR}[1]{\FRM{\,#1\,}}
%\newcommand{\FRM}[1]{\mbox{$#1$}}
%\newcommand{\DERIVE}
%  {\setlength{\baselineskip}{1.3\baselineskip}
%  \begin{tabbing}
%  \hspace{\OPAR}\=\hspace{\OPAR}\=\hspace{\OPAR}\=\hspace{\OPAR}\=\hspace{\OPAR}\=\hspace{\OPAR}\=\kill
%  }
%\newcommand{\form}[1]
%  {\>\>\FOR{#1}}
%\newcommand{\hint}[2]
%  {\\*\>\FRM{#1}\>\>\{#2\}\\}
%\newcommand{\STARTPROOF}[1]
%  {\TOPROVE{#1}\\$\vartriangleright$ \>\>}
%\newcommand{\TOPROVE}[1]
%  {#1}
%\newcommand{\ENDDERIVE}
%  {\end{tabbing}\setlength{\baselineskip}{\OBLS}
%}
%\newcommand{\ENDPROOF}
%  {$\rbrack\arrowvert$\\
%  $\square$\\\\}
%\newcommand {\IND}
%  {\setlength{\parindent}{\OPAR}\\
%  \begin{indent}}
%\newcommand {\ENDIND}
%  {\end{indent}\\
%  \setlength{\parindent}{0 pt}\\}
%\newcommand {\IF}
%  {\mathbf{if} \:\:}
%\newcommand {\BAR}
%  {\ [\!] \:\:\:}
%\newcommand {\FI}
%  {\mathbf{fi} \:\:}
%\newcommand {\ARROW}
%  {\rightarrow}
%\newcommand{\clearemptydoublepage}{\newpage{\pagestyle{empty}\cleardoublepage}}
%\newcommand{\TT} {\textit{true}}
%\newcommand{\FF} {\textit{false}}


%%%%%%%%%%%%%%%%%%%% MACROS %%%%%%%%%%%%%%%%%%%%
\floatstyle{plain}
\newfloat{grammar}{thp}{lop}
\floatname{grammar}{Grammar}


%%%%%%%%%%%%%%%%%%%%%Stuff from SICL article%%%%%%
%\newcommand{\sos}[1]{\mathcal{#1}}
%\newcommand{\sset}[1]{\texttt{#1}}
%\newcommand{\restr}[2]{#1\restriction#2}

%\def\universe{A}
%\def\Empty{\mbox{\O}}





%%%%%%%%%%%%%%%%%%% DOCUMENT %%%%%%%%%%%%%%%%%%%
\begin{document}

\title{A case for the use of delayed reductions in left-to-right parsing of context-free grammars}
\author{Rehno Lindeque
\\23123576}

\maketitle

\begin{abstract}
In this paper we suggest the use of delayed actions in generalized parsing of context-free grammars. 
Our contention is that delayed reductions in a left-to-right parser could be used to implement more efficient parsers for unambiguous context-free grammars. 
This work is therefore intended as preparatory material for further research in this area. 
To this end we develop a set of tools for studying and developing the semantics of delayed parsing actions.
A small pseudo instruction set is shown which includes delayed parsing actions in order to resolve rules containing cycles in their lookahead sets.
Operational semantics for each of these instructions are described and the implementation of an interpreter for the instructions is demonstrated.
Finally we present a prototype construction algorithm capable of generating parsers for a limited subset of unambiguous grammars.\\
\end{abstract}

\textbf{Keywords:} LD parsing, context-free grammar, generalized parsing, GLR parsing, LR parsing, LL parsing

\section{Introduction}

\subsection{Top-down vs Bottom-up parsing}


\subsection{Generalized parsing approaches}

\section{Delayed reductions}

\subsection{Basic strategy}
Our basic strategy for parsing grammars proceeds in two passes.
The first is a recognition pass where the sequence of reduction rules are recorded into a temporary buffer that grows in a stack-like manner and supports constant-time random-access.
In the second pass the sequence of recognized rules are applied consecutively in order to build the final parse tree in a manner similar to LR parser, but without any recognition steps or lookahead.
By separating these two steps into distinct passes, the recognition pass is allowed the convenience of identifying rules in a more flexible sequence.\\

The mechanism used to reorder rules in this paper is placeholder tokens representing delayed reductions.
Whenever a rule must be reduced that cannot be fully determined, a placeholder token is recorded in the buffer in its place. 
Once the reduction rule can be resolved the placeholder token is replaced with this rule number.
In order to keep track of placeholders we also record their indexes in a separate stack, allowing the parser to return to these tokens later on in order to replace them using random-access writes.\\

In effect placeholders are used in order delay reductions until some unspecified lookahead has been recognized.
Clearly the order of rule recognition is relaxed using this mechanism, however it is not completely unrestricted. 
The following conditions specify certain constraints by which the algorithm must abide.

\begin{itemize}
\item As rules are recognized they are placed in the buffer in a left-to-right order. This corresponds with a bottom-up parsing strategy since productions that are closer to the leaf nodes in the parse tree will normally be recognized before their parent rules are recognized.
\item Delayed rules may only be resolved in a right-to-left order, corresponding with the top-down strategy of parsing since delayed rules near the root of the tree must be resolved before 
rules near the leaf nodes can be resolved. 
It is worth noting however that, although the unresolved rules will only be resolved in a right-to-left order, rules may be added to the unresolved stack at any time which intuitively
causes the right-to-left rule resolution loop to be nested inside the main left-to-right recognition loop.
\end{itemize}

In order to explain this strategy in more formal terms we will define an instruction set and interpreter which in essence forms a tiny domain specific language for parsing with delayed actions.
For the sake of simplicity we begin by describing a simple modified LR(0) algorithm. This LR(0) algorithm has three basic operations: shift, reduce and pivot. 
The shift and pivot operation is responsible for recognizing terminal tokens on the input stream whereas the reduce action is responsible for translating strings of terminal tokens into nonterminal tokens.

An interpreter can be built to correctly execute these instructions. 
Using this approach we can demonstrate the operation of a parser constructed using this set of actions.
The following free variables will be used to represent the state of the parser interpreter during execution:

\begin{itemize}
\item $input_{[0, input.length)}$ is an array of input tokens of length $input.length$ returned by a lexer.
\item $i$ is an index into the array of input tokens representing the current position of the parser in the stream during the recognition pass such that $input_i$ is the token currently being tested.
This index will never be decreased and need only increase in increments of 1.
\item $actions_{[0, actions.length)}$ is a sequence of instructions representing the encoded parser similar to the source code that is output by known parser generators. 
This array is automatically generated by a separate parser construction algorithm which will be discussed later.
\item $j$ is an index into the array of actions such that $actions_j$ defines the current instruction being decoded.
\item $outputrules$ is dynamically growing array storing the output of recognition pass
\item $error$ is a special error state which (in the absence of error recovery) will halt the parser and report a syntax error.
\end{itemize}

Every action in the array will be represented using the pattern $action([parameters...])$.
(todo: determine the correct syntax for this notation)\\

Then the operational semantics of these three operations is given as follows:\\

\texttt{shift(terminaltoken)}\\
The shift action is used to step over input tokens deterministicly. 
There are only two options: Either the token matches the parameter given to the shift action, or the parser is put into an error state.

\begin{equation}
\infer{i, j \mapsto i+1, j+1}{input_i = a & actions_j = shift(a)} \tag{Def 1.1}
\end{equation}

\begin{equation}
\infer{error}{input_i = a & actions_j = shift(b)} \tag{Def 1.2}
\end{equation}\\

\texttt{pivot([terminaltoken $\Rightarrow$ target, ...])}\\
The pivot action is used to recognize input tokens nondeterministicly. 
There is an apparent redundancy in the difference between the pivot and the shift action. [todo]. 
We can see that a shift action in the form $shift(a)$ will behave similar to a pivot action of the form $pivot([a \Rightarrow k])$ where k indicates the index of the next action to perform.
However a distinction is made between the two actions as a matter of convenience:
Pivot actions are in principle the central recognition mechanism of the parser. 
In contrast, shift actions does not make any significant contribution to the internal state of the parser other than to catch syntax errors.

\begin{equation}
\infer{i, j \mapsto i+1, gettarget(A, input_i)}{input_i \in A & actions_j = pivot(A)} \tag{Def 1.3}
\end{equation}

\begin{equation}
\infer{error}{input_i \notin A & actions_j = pivot(A)} \tag{Def 1.4}
\end{equation}\\

\texttt{reduce(rule)}\\
The parameter given to the reduce action is an index into a list of all rules in the given grammar where every rule is in the form \texttt{nonterminaltoken $\rightarrow$ terminaltoken ...}. 
A production in a grammar may be reduced from any number of different sequences of terminal tokens, however a rule recognizes only one possible sequence of terminal tokens.
In other words, given some rule both the left and right side of the reduction can be determined (although identifiers must still be stored).
For this reason we use rule numbers internally rather than using productions directly.

\begin{equation}
\infer{j, outputrules_k, k \mapsto j+1, r, k+1}{input_i \in A & actions_j = reduce(r)} \tag{Def 1.5}
\end{equation}\\


\subsubsection{todo: busy here... Basic LR(0) grammar using these instructions should be demonstrated}

To see how this program translates to a familiar language we show the corresponding in the guarded command language (GCL).

\begin{figure}[htbp]
\begin{center}
\begin{gcl}
\PROC lr0parse()
%\PRE \{sorted(A) \land sorted(B) \land sorted(C)\}
%i, j, k \becomes 0, 0, 0;
%\textbf{Invariant: } \{ inv \triangleq sorted(D_{[0, r)}) \land D_{[0, r)} = A_{[0, i)} \cap B_{[0, j)} \cap C_{[0, k)} \}
%\textbf{Variant: }  \{  var \triangleq i \in [0, A.len) \land j \in [0, B.len) \land k \in [0, C.len) \}
%;\DO (var \land A_i = A_j = A_k) \rightarrow
%\ \ \     D_r \becomes A_i;
%\ \ \     i, j, k, r \becomes i + 1, j + 1, k + 1, r + 1
%\BAR (var \land A_i \leq A_j \land A_i \leq A_k) \rightarrow i \becomes i + 1
%\BAR (var \land A_j \leq A_i \land A_j \leq A_k) \rightarrow j \becomes j + 1
%\BAR (var \land A_k \leq A_i \land A_k \leq A_j) \rightarrow k \becomes k + 1
%\OD
%\textbf{Invariant} \land \neg var
%\POST \{sorted(D) \land D = A \cap B \cap C\}
\CORP
\end{gcl}
\caption{An LR(0) parser example translated to GCL}
%\label{GCLKourie}
\end{center}
\end{figure}


\subsection{Delayed actions}
While the four operations discussed provide the necessary tools to recognize any LR(0) grammar, they do not provide any manner of looking ahead into the input stream.
Suppose that we add two more operations, \texttt{delay()} and \texttt{resolve(rule)} to our instruction set with the following semantics:\\

\texttt{delay()}\\
The delay action adds a placeholder token called $ignore$ to the output rules which may be replaced later.
In addition the index of the placeholder is pushed onto the stack of delays.
Note that the value of $ignore$ should be reserved so that it will not conflict with any rule number.
\begin{equation}
\infer{delays_l, l, j, outputrules_k, k \mapsto j, l+1, j+1, ignore, k+1}{actions_j = delay()} \tag{Def 1.6}
\end{equation}\\

\texttt{resolve(rule)}\\
The resolve action is similar to the reduce action but in this 
\begin{equation}
\infer{l, j, outputrules_{delays_l} \mapsto l-1, j+1, ignore}{actions_j = resolve(r)} \tag{Def 1.7}
\end{equation}\\

This pair works together in order to implement \emph{delayed reductions}. 
It is easy to see that the action $delay()$ immediately followed by the action $resolve(r)$ is identical to our original action $reduce(r)$. 
In the remainder of this section we will show how lookahead can be implemented by placing other actions between these two actions.

However in order for this pair to work correctly we must also modify our original pivot and return actions.
\begin{equation}
\infer{j, m \mapsto continuations_m, m-1}{actions_j = return()} \tag{Def 1.8}
\end{equation}\\

\begin{equation}
\infer{i, j \mapsto i+1, gettarget(A, input_i)}{input_i \in A & actions_j = pivot(A)} \tag{Def 1.9}
\end{equation}\\


\subsection{Traces}
To see how the delay / resolve pair can be used to look ahead in the we must investigate several different forms of grammar rules.
In order to do this we must generate all possible input strings for each form. 
The concept of traces are used to show the actions that the parser must take for possibility.
A trace is simply a sequence of actions that the parser must take to recognize a particular instance of a valid input stream for the given grammar.
A mechanical method of generating traces is used such that every possibility can be enumerated, although the trace itself may not be finite.\\

In the following examples capitalized tokens represent nonterminals with subscripts differentiating between nonterminals produced by different rules. Lowercase tokens represent terminals and $S$ is always the starting nonterminal.\\

%%%%%%%%%%%%% Example 1: Simple grammar
\subsubsection{Reduction based on a fixed lookahead}
\begin{tabular}[t]{cl}
%\begin{grammar}
%\begin{wrapfigure}{l}{.2\textwidth}
\parbox{.3\textwidth}{
\begin{align*}
G_1 \equiv \quad & A_1 \rightarrow x\\
                 & B_1 \rightarrow x\\
                 & S_1 \rightarrow A a \$\\
                 & S_2 \rightarrow B b \$
\end{align*}}
%\end{wrapfigure}
%\end{grammar}
\parbox{.8\textwidth}{This is the simplest grammar requiring lookahead. Its traces can be written as:}
\end{tabular}

\begin{align*}
traces(G_1) \equiv \{ & \textless shift(x), reduce(A_1), pivot(a), reduce(S_1) \textgreater,\\
                      & \textless shift(x), pivot(b), reduce(S_2) \textgreater \}
\end{align*}

However, note that this trace also be expanded with delay actions, so that the lookahead could be implemented. Hence an equivalent trace to the above is:
\begin{align*}
traces(G_1) \equiv \{ & \textless shift(x), pivot(a), reduce(A), reduce(S_1) \textgreater, \textless shift(x), pivot(b), reduce(S_2) \textgreater \}
\end{align*}

%%%%%%%%%%%%% Example 2: Optional reduction based on a fixed lookahead
\subsubsection{Optional reduction based on a fixed lookahead}
\begin{tabular}[t]{cl}
\parbox{.3\textwidth}{
\begin{align*}
G_2 \equiv \quad & A_1 \rightarrow x\\
                 & S_1 \rightarrow A a \$\\
                 & S_2 \rightarrow x b \$
\end{align*}}
\parbox{.8\textwidth}{Here is a simple grammar requiring lookahead, but where some reductions do not take place.}
\end{tabular}

\begin{align*}
traces(G_2) \equiv \{ & \textless shift(x), reduce(A_1), pivot(a), reduce(S_1) \textgreater,\\
                      & \textless shift(x), pivot(b), reduce(S_2) \textgreater \}
\end{align*}

The equivalent trace with delays can be written as follows:
\begin{align*}
traces(G_2) \equiv \{ & \textless shift(x), pivot(a), reduce(A_1), reduce(S_1) \textgreater,\\
                      & \textless shift(x), pivot(b), reduce(S_2) \textgreater \}
\end{align*}

%%%%%%%%%%%%% Example 3: Disjointed reductions that require lookahead
\subsubsection{Disjointed reductions that require lookahead}
\begin{tabular}[t]{cl}
\parbox{.3\textwidth}{
\begin{align*}
G_3 \equiv \quad & A_1 \rightarrow x y\\
                 & B_1 \rightarrow y b\\
                 & C_1 \rightarrow x\\
                 & S_1 \rightarrow A a \$\\
                 & S_2 \rightarrow C B \$
\end{align*}}
\parbox{.8\textwidth}{sfd}
\end{tabular}

\begin{align*}
traces(G_3) \equiv \{ & \textless shift(x), shift(y), reduce(A_1), pivot(a), reduce(S_1) \textgreater,\\
                      & \textless shift(x), reduce(C_1), shift(y), pivot(b), reduce(B_1), reduce(S_2) \textgreater \}
\end{align*}

The equivalent trace with delays can be written as follows:
\begin{align*}
traces(G_3) \equiv \{ & \textless shift(x), shift(y), pivot(a), reduce(A_1), reduce(S_1) \textgreater,\\
                      & \textless shift(x), shift(y), pivot(b), reduce(C_1), reduce(B), reduce(S_2) \textgreater \}
\end{align*}

%%%%%%%%%%%%% Example 4: Right recursive grammars
\subsubsection{Right recursive grammars}
\begin{tabular}[t]{cl}
\parbox{.3\textwidth}{
\begin{align*}
G_4 \equiv \quad & A_1 \rightarrow x\\
                 & A_2 \rightarrow x A\\
                 & B_1 \rightarrow x\\
                 & B_2 \rightarrow x B\\
                 & S_1 \rightarrow A a \$\\
                 & S_2 \rightarrow B b \$
\end{align*}}
\parbox{.8\textwidth}{sfd}
\end{tabular}

\begin{align*}
traces(G_4) \equiv \{ & \textless shift(x), reduce(A_1), pivot(a), reduce(S_1) \textgreater,\\
                      & \textless shift(x), shift(x), reduce(A_1), reduce(A_2), pivot(a), reduce(S_1) \textgreater, ... \\
                      & \textless shift(x), reduce(B_1), pivot(b), reduce(S_2) \textgreater,\\
                      & \textless shift(x), shift(x), reduce(B_1), reduce(B_2), pivot(b), reduce(S_2) \textgreater, ... \}
\end{align*}

The equivalent trace with delays can be written as follows:
\begin{align*}
traces(G_4) \equiv \{ & \textless shift(x), delay, pivot(a), resolve(A_1), reduce(S_1) \textgreater,\\
                      & \textless shift(x), shift(x), delay, delay, pivot(a), resolve(A_2), resolve(A_1), reduce(S_1) \textgreater, ... \\
                      & \textless shift(x), delay, pivot(b), resolve(B_1), reduce(S_2) \textgreater,\\
                      & \textless shift(x), shift(x), delay, delay, pivot(b), resolve(B_2), resolve(B_1), reduce(S_2) \textgreater, ... \}
\end{align*}

%%%%%%%%%%%%% Example 5: Combined left and right recursive grammar
\subsubsection{Combined left and right recursive grammar}
\begin{tabular}[t]{cl}
\parbox{.3\textwidth}{
\begin{align*}
G_5 \equiv \quad & A_1 \rightarrow x\\
                 & A_2 \rightarrow x A\\
                 & B_1 \rightarrow x\\
                 & B_2 \rightarrow B x\\
                 & S_1 \rightarrow A a \$\\
                 & S_2 \rightarrow B b \$
\end{align*}}
\parbox{.8\textwidth}{sfd}
\end{tabular}

\begin{align*}
traces(G_5) \equiv \{ & \textless shift(x), delay, pivot(a), resolve(A_1), reduce(S_1) \textgreater,\\
                      & \textless shift(x), shift(x), delay, delay, pivot(a), resolve(A_2), resolve(A_1), reduce(S_1) \textgreater, ... \\
                      & \textless shift(x), delay, pivot(b), resolve(B_1), reduce(S_2) \textgreater,\\
                      & \textless shift(x), shift(x), delay, delay, pivot(b), resolve(B_2), resolve(B_1), reduce(S_2) \textgreater, ... \}
\end{align*}

The equivalent trace with delays can be written as follows:
\begin{align*}
traces(G_5) \equiv \{ & \textless shift(x), delay, pivot(a), resolve(A_1), reduce(S_1) \textgreater,\\
                      & \textless shift(x), shift(x), delay, delay, pivot(a), resolve(A_2), resolve(A_1), reduce(S_1) \textgreater, ... \\
                      & \textless shift(x), delay, pivot(b), resolve(B_1), reduce(S_2) \textgreater,\\
                      & \textless shift(x), shift(x), delay, delay, pivot(b), resolve(B_2), resolve(B_1), reduce(S_2) \textgreater, ... \}
\end{align*}

%%%%%%%%%%%%% Example 6: 
\subsubsection{}
\begin{tabular}[t]{cl}
\parbox{.3\textwidth}{
\begin{align*}
G_6 \equiv \quad & A_1 \rightarrow x\\
                 & B_1 \rightarrow x\\
                 & C_1 \rightarrow y\\
                 & C_2 \rightarrow A C a\\
                 & C_3 \rightarrow B C b\\
                 & S_1 \rightarrow C \$
\end{align*}}
\parbox{.8\textwidth}{sfd}
\end{tabular}

\begin{align*}
traces(G_6) \equiv \\
\{ & \textless pivot(y), reduce(C_1), shift(\$), reduce(S_1) \textgreater,\\
                      & \textless pivot(x), reduce(A_1), pivot(y), reduce(C_1), pivot(a), reduce(C_2), shift(\$), reduce(S_1) \textgreater,\\
                      & \textless pivot(x), reduce(B_1), pivot(y), reduce(C_1), pivot(b), reduce(C_3), shift(\$), reduce(S_1) \textgreater,\\
                      & \textless pivot(x), reduce(A_1), pivot(x), reduce(A_1), pivot(y), reduce(C_1), pivot(a), reduce(C_2), pivot(a), reduce(C_2), shift(\$), reduce(S_1) \textgreater,\\
                      & \textless pivot(x), reduce(A_1), pivot(x), reduce(B_1), pivot(y), reduce(C_1), pivot(b), reduce(C_3), pivot(a), reduce(C_2), shift(\$), reduce(S_1) \textgreater,\\
                      & \textless pivot(x), reduce(B_1), pivot(x), reduce(A_1), pivot(y), reduce(C_1), pivot(a), reduce(C_2), pivot(b), reduce(C_3), shift(\$), reduce(S_1) \textgreater,\\
                      & \textless pivot(x), reduce(B_1), pivot(x), reduce(B_1), pivot(y), reduce(C_1), pivot(b), reduce(C_3), pivot(b), reduce(C_3), shift(\$), reduce(S_1) \textgreater,\\
                      & ... \}
\end{align*}

The equivalent trace with delays can be written as follows:
\begin{align*}
traces(G_6) \equiv \\
\{ & \textless pivot(y), reduce(C_1), shift(\$), reduce(S_1) \textgreater,\\
                      & \textless pivot(x), delay(A,B), pivot(y), reduce(C_1), pivot(a), resolve(A), reduce(C_2), shift(\$), reduce(S_1) \textgreater,\\
                      & \textless pivot(x), delay(A,B), pivot(y), reduce(C_1), pivot(b), resolve(B), reduce(C_3), shift(\$), reduce(S_1) \textgreater,\\
                      & \textless pivot(x), delay(A,B), pivot(x), delay(A,B), pivot(y), reduce(C_1), pivot(a), resolve(A), reduce(C_2), pivot(a), resolve(A), reduce(C_2), shift(\$), reduce(S_1) \textgreater,\\
                      & \textless pivot(x), delay(A,B), pivot(x), delay(A,B), pivot(y), reduce(C_1), pivot(b), resolve(B), reduce(C_3), pivot(a), resolve(A), reduce(C_2), shift(\$), reduce(S_1) \textgreater,\\
                      & \textless pivot(x), delay(A,B), pivot(x), delay(A,B), pivot(y), reduce(C_1), pivot(a), resolve(A), reduce(C_2), pivot(b), resolve(B), reduce(C_3), shift(\$), reduce(S_1) \textgreater,\\
                      & \textless pivot(x), delay(A,B), pivot(x), delay(A,B), pivot(y), reduce(C_1), pivot(b), resolve(B), reduce(C_3), pivot(b), resolve(B), reduce(C_3), shift(\$), reduce(S_1) \textgreater,\\
                      & ... \}
\end{align*}

%%%%%%%%%%%%% Example 7: 
\subsubsection{}
\begin{tabular}[t]{cl}
\parbox{.3\textwidth}{
\begin{align*}
G_7 \equiv \quad & A_1 \rightarrow x\\
                 & B_1 \rightarrow x\\
                 & C_1 \rightarrow y\\
                 & C_2 \rightarrow C A a\\
                 & C_3 \rightarrow C B b\\
                 & S_1 \rightarrow C \$
\end{align*}}
\parbox{.8\textwidth}{sfd}
\end{tabular}

\begin{align*}
traces(G_7) \equiv \\
\{ & \textless shift(y), reduce(C_1), pivot(\$), reduce(S_1) \textgreater,\\
                      & \textless shift(y), reduce(C_1), pivot(x), reduce(A_1), pivot(a), reduce(C_2), pivot(\$), reduce(S_1) \textgreater,\\
                      & \textless shift(y), reduce(C_1), pivot(x), reduce(B_1), pivot(b), reduce(C_3), pivot(\$), reduce(S_1) \textgreater,\\
                      & \textless shift(y), reduce(C_1), pivot(x), reduce(A_1), pivot(a), reduce(C_2), pivot(x), reduce(A_1), pivot(a), reduce(C_2), pivot(\$), reduce(S_1) \textgreater,\\
                      & \textless shift(y), reduce(C_1), pivot(x), reduce(B_1), pivot(b), reduce(C_3), pivot(x), reduce(A_1), pivot(a), reduce(C_2), pivot(\$), reduce(S_1) \textgreater,\\
                      & \textless shift(y), reduce(C_1), pivot(x), reduce(A_1), pivot(a), reduce(C_2), pivot(x), reduce(B_1), pivot(b), reduce(C_3), pivot(\$), reduce(S_1) \textgreater,\\
                      & \textless shift(y), reduce(C_1), pivot(x), reduce(B_1), pivot(b), reduce(C_3), pivot(x), reduce(B_1), pivot(b), reduce(C_3), pivot(\$), reduce(S_1) \textgreater,\\
                      & ... \}
\end{align*}

The equivalent trace with delays can be written as follows:
\begin{align*}
traces(G_7) \equiv \\
\{ & \textless shift(y), reduce(C_1), pivot(\$), reduce(S_1) \textgreater,\\
                      & \textless shift(y), reduce(C_1), pivot(x), delay(A,B), pivot(a), resolve(A_1), reduce(C_2), pivot(\$), reduce(S_1) \textgreater,\\
                      & \textless shift(y), reduce(C_1), pivot(x), delay(A,B), pivot(b), resolve(B_1), reduce(C_3), pivot(\$), reduce(S_1) \textgreater,\\
                      & \textless shift(y), reduce(C_1), pivot(x), delay(A,B), pivot(a), resolve(A_1), reduce(C_2), pivot(x), delay(A,B), pivot(a), resolve(A_1), reduce(C_2), pivot(\$), reduce(S_1) \textgreater,\\
                      & \textless shift(y), reduce(C_1), pivot(x), delay(A,B), pivot(a), resolve(A_1), reduce(C_2), pivot(x), delay(A,B), pivot(b), resolve(B_1), reduce(C_3), pivot(\$), reduce(S_1) \textgreater,\\
                      & \textless shift(y), reduce(C_1), pivot(x), delay(A,B), pivot(b), resolve(B_1), reduce(C_3), pivot(x), delay(A,B), pivot(b), resolve(B_1), reduce(C_3), pivot(\$), reduce(S_1) \textgreater,\\
                      & \textless shift(y), reduce(C_1), pivot(x), delay(A,B), pivot(b), resolve(B_1), reduce(C_3), pivot(x), delay(A,B), pivot(a), resolve(A_1), reduce(C_2), pivot(\$), reduce(S_1) \textgreater,\\
                      & ... \}
\end{align*}

%%%%%%%%%%%%% Example 8: 
\subsubsection{}
\begin{tabular}[t]{cl}
\parbox{.3\textwidth}{
\begin{align*}
G_8 \equiv \quad & A_1 \rightarrow x\\
                 & B_1 \rightarrow x\\
                 & C_1 \rightarrow y\\
                 & C_2 \rightarrow A C a\\
                 & C_3 \rightarrow C B b\\
                 & S_1 \rightarrow C \$
\end{align*}}
\parbox{.8\textwidth}{sfd}
\end{tabular}

\begin{align*}
traces(G_8) \equiv \\
\{ & \textless pivot(y), reduce(C_1), pivot(\$), reduce(S_1) \textgreater,\\
                      & \textless pivot(x), reduce(A_1), pivot(y), reduce(C_1), pivot(a), reduce(C_2), pivot(\$), reduce(S_1) \textgreater,\\
                      & \textless pivot(y), reduce(C_1), pivot(x), reduce(B_1), shift(b), reduce(C_3), pivot(\$), reduce(S_1) \textgreater,\\
                      & \textless pivot(x), reduce(A_1), pivot(x), reduce(A_1), pivot(y), reduce(C_1), shift(a), reduce(C_2), shift(a), reduce(C_2), pivot(\$), reduce(S_1) \textgreater,\\
                      & \textless pivot(x), reduce(A_1), pivot(y), reduce(C_1), pivot(x), reduce(B_1), pivot(b), reduce(C_3), shift(a), reduce(C_2), pivot(\$), reduce(S_1) \textgreater,\\
                      & \textless pivot(x), reduce(A_1), pivot(y), reduce(C_1), pivot(a), reduce(C_2), pivot(x), reduce(B_1), shift(b), reduce(C_3), pivot(\$), reduce(S_1) \textgreater,\\
                      & \textless pivot(y), reduce(C_1), pivot(x), reduce(B_1), shift(b), reduce(C_3), pivot(x), reduce(B_1), shift(b), reduce(C_3), pivot(\$), reduce(S_1) \textgreater,\\
                      & ... \}
\end{align*}

The equivalent trace with delays can be written as follows:
\begin{align*}
traces(G_8) \equiv \\
\{ & \textless pivot(y), reduce(C_1), pivot(\$), reduce(S_1) \textgreater,\\
                      & \textless pivot(x), reduce(A_1), pivot(y), reduce(C_1), pivot(a), reduce(C_2), pivot(\$), reduce(S_1) \textgreater,\\
                      & \textless pivot(y), reduce(C_1), pivot(x), reduce(B_1), shift(b), reduce(C_3), pivot(\$), reduce(S_1) \textgreater,\\
                      & \textless pivot(x), reduce(A_1), pivot(x), reduce(A_1), pivot(y), reduce(C_1), shift(a), reduce(C_2), shift(a), reduce(C_2), pivot(\$), reduce(S_1) \textgreater,\\
                      & \textless pivot(x), reduce(A_1), pivot(y), reduce(C_1), pivot(x), reduce(B_1), pivot(b), reduce(C_3), shift(a), reduce(C_2), pivot(\$), reduce(S_1) \textgreater,\\
                      & \textless pivot(x), reduce(A_1), pivot(y), reduce(C_1), pivot(a), reduce(C_2), pivot(x), reduce(B_1), shift(b), reduce(C_3), pivot(\$), reduce(S_1) \textgreater,\\
                      & \textless pivot(y), reduce(C_1), pivot(x), reduce(B_1), shift(b), reduce(C_3), pivot(x), reduce(B_1), shift(b), reduce(C_3), pivot(\$), reduce(S_1) \textgreater,\\
                      & ... \}
\end{align*}

%S1{ C1{y} $ }
%S1{ C2{ A{x} C1{y} a} $ }
%S1{ C3{ C1{y} B{x} b } $ }
%S1{ C2{ A{x} C2{ A{x} C1{y} a } a } $ }
%S1{ C2{ A{x} C3{ C1{y} B{x} b } a } $ }
%S1{ C3{ C2{ A{x} C1{y} a } B{x} b } $ }
%S1{ C3{ C3{ C1{y} B{x} b } B{x} b } $ }

%(Surprisingly enough, this one needs no look-aheads)


%%%%%%%%%%%%% Example 9: A delayed rule in front of a cycle
\subsubsection{A delayed rule in front of a cycle}
\begin{tabular}[t]{cl}
\parbox{.3\textwidth}{
\begin{align*}
G_9 \equiv \quad & A_1 \rightarrow x\\
                 & B_1 \rightarrow x\\
                 & C_1 \rightarrow y\\
                 & C_2 \rightarrow C y\\
                 & S_1 \rightarrow A C a \$\\
                 & S_2 \rightarrow B C b \$
\end{align*}}
\parbox{.8\textwidth}{sfd}
\end{tabular}

\begin{align*}
traces(G_9) \equiv \\
\{ & \textless shift(x), reduce(A_1), shift(y), reduce(C_1), pivot(a), shift(\$), reduce(S_1) \textgreater,\\
                      & \textless shift(x), reduce(B_1), shift(y), reduce(C_1), pivot(b), shift(\$), reduce(S_2) \textgreater,\\
                      & \textless shift(x), reduce(A_1), shift(y), reduce(C_1), pivot(y), reduce(C_2), pivot(a), shift(\$), reduce(S_1) \textgreater,\\
                      & \textless shift(x), reduce(B_1), shift(y), reduce(C_1), pivot(y), reduce(C_2), pivot(b), shift(\$), reduce(S_2) \textgreater,\\
                      & ... \}
\end{align*}

The equivalent trace with delays can be written as follows:
\begin{align*}
traces(G_9) \equiv \\
\{ & \textless shift(x), delay(A,B), shift(y), reduce(C_1), pivot(a), resolve(A_1), shift(\$), reduce(S_1) \textgreater,\\
                      & \textless shift(x), delay(A,B), shift(y), reduce(C_1), pivot(b), resolve(B_1), shift(\$), reduce(S_2) \textgreater,\\
                      & \textless shift(x), delay(A,B), shift(y), reduce(C_1), pivot(y), reduce(C_2), pivot(a), resolve(A), shift(\$), reduce(S_1) \textgreater,\\
                      & \textless shift(x), delay(A,B), shift(y), reduce(C_1), pivot(y), reduce(C_2), pivot(b), resolve(B), shift(\$), reduce(S_2) \textgreater,\\
                      & ... \}
\end{align*}

%S1 { A{x} C1{y} a }
%S2 { B{x} C1{y} b }
%S1 { A{x} C2{ C1{y} y } a }
%S2 { B{x} C2{ C1{y} y } b }


%%%%%%%%%%%%% Example 10: (Looking ahead into a cycle)
\subsubsection{(Looking ahead into a cycle)}
\begin{tabular}[t]{cl}
\parbox{.3\textwidth}{
\begin{align*}
G_10 \equiv \quad & A_1 \rightarrow x\\
                  & B_1 \rightarrow x\\
                  & C_1 \rightarrow a\\
                  & C_2 \rightarrow y C\\
                  & D_1 \rightarrow b\\
                  & D_2 \rightarrow y C\\
                  & S_1 \rightarrow A C \$\\
                  & S_2 \rightarrow B D \$
\end{align*}}
\parbox{.8\textwidth}{sfd}
\end{tabular}

\begin{align*}
traces(G_10) \equiv \\
\{ & \textless shift(x), reduce(A_1), pivot(a), reduce(C_1), shift(\$), reduce(S_1) \textgreater,\\
                       & \textless shift(x), reduce(B_1), pivot(b), reduce(D_1), shift(\$), reduce(S_2) \textgreater,\\
                       & \textless shift(x), reduce(A_1), pivot(y), pivot(a), reduce(C_1), reduce(C_2), shift(\$), reduce(S_1) \textgreater,\\
                       & \textless shift(x), reduce(B_1), pivot(y), pivot(b), reduce(D_1), reduce(D_2), shift(\$), reduce(S_2) \textgreater,\\
                       & ... \}
\end{align*}

The equivalent trace with delays can be written as follows:
\begin{align*}
traces(G_10) \equiv \\
 \{ & \textless shift(x), delay(A, B), pivot(a), reduce(C_1), resolve(A), shift(\$), reduce(S_1) \textgreater,\\
    & \textless shift(x), delay(A, B), pivot(b), reduce(D_1), resolve(B), shift(\$), reduce(S_2) \textgreater,\\
    & \textless shift(x), delay(A, B), pivot(y), pivot(a), reduce(C_1), reduce(C_2), resolve(A), shift(\$), reduce(S_1) \textgreater,\\
    & \textless shift(x), delay(A, B), pivot(y), pivot(b), reduce(D_1), reduce(D_2), resolve(B), shift(\$), reduce(S_2) \textgreater,\\
    & ... \}
\end{align*}

%S1 { A{x} C1{a} $ }
%S2 { B{x} D1{b} $ }
%S1 { A{x} C2{ y C1{a} } $ }
%S2 { B{x} D2{ y D1{b} } $ }

%%%%%%%%%%%%%%%%%%% IMPLEMENTATION
\section{Implementation}
\subsection{Parser construction}
\subsection{Parser interpreter}

\section*{Results}

\section*{Conclusions}

\section*{Future work}

Combining the recognition and builder passes.
The cost of Amortized parallelization.

\end{document}

