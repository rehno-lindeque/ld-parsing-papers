\documentclass[envcountsame,runningheads]{llncs}
\usepackage{pscproc2}
\usepackage{url}


% This is the shorter version of the technical report on "left-to-right parsing with delayed reductions" for the Prague Stringology Conference
\begin{document}
\title{Left-to-right parsing with delayed reductions}
\author{Rehno Lindeque \and Derrick Kourie}
\institute{University of Pretoria\\
\email{research@rehno.lindeque.name}}
\maketitle

\begin{abstract}
In this paper we suggest the use of delayed actions for parsing context-free languages.
We develop tools to study and develop the semantics of a domain specific language for building parsers with delayed reductions.
An informal proof shows that a parser implemented using our semantics will perform in linear time with respect to the number of lexical tokens.
Finally a prototype parser generator is described, capable of automatically generating parsers for a subset of context-free grammars.
\end{abstract}

\begin{keywords}
LD parser, recursive-ascent parser, context-free grammar, parser generator, bottom-up parser, top-down parser, delayed reductions
\end{keywords}

\section{Introduction}


\section{Results}

\section{Conclusions}
An earlier technical report describing additional details of the parsing DSL and parser generator can be found in \cite{Lin11}.

\section{Future work}

\bibliographystyle{psc}
\bibliography{ldparsing}

\end{document}

