
\documentclass[11pt]{article}

%%%%%%%%%%%%%%%%%%% PACKAGES %%%%%%%%%%%%%%%%%%%
\usepackage{geometry}
\geometry{a4paper}
%\usepackage{gcl}
%\usepackage{pfs}
%\usepackage{algorithmic}
%\usepackage{graphicx}
%\usepackage{epstopdf}
%\usepackage{natbib}
%\usepackage{fancyvrb}
%\usepackage{float}
%\usepackage{wrapfig}
%\usepackage{proof}

\usepackage{amssymb}
\usepackage{amsthm}
\usepackage{amsmath}
\usepackage{stmaryrd}


%%%%%%%%%%%%%%%%%%%% MACROS %%%%%%%%%%%%%%%%%%%%

%%%%%%%%%%%%%%%%%%%%%Stuff from SICL article%%%%%%

%%%%%%%%%%%%%%%%%%% DOCUMENT %%%%%%%%%%%%%%%%%%%
\begin{document}

%\title{A formal foundation for programming language design}
\title{Simple programs\\\small{(A formal foundation for programming languages)}}
\author{Rehno Lindeque}

\maketitle

\begin{abstract}
A considerable amount of research effort has been vested in exploring the relationship between computer programming languages and well-defined formal languages such as mathematics and logic.
This paper attempts to consolidate various mathematical theories and semantic models into a unified model of computation that is both simple and practical.
We take care to model concepts that are not typically described in pure mathematics but which are pervasive in many real-world computer programs.
In particular non-determinism, explicit processes and constraint programming is included in the formal system.


Our observation that both mathematical and computer programming languages are self-referential leads us to separate computational models into two orthogonal yet reciprocal concepts that is based on the Curry-Howard correspondence.
Finally we study our model using concepts from abstract and foundational mathematics such as category theory.
Hence programming expressions are recast in terms of `definitions' and `queries' which forms the basis of our paradigm.
%Hence, we reframe various computational theories in terms of `definitions' and `queries' by drawing on .

%In this paper we explore advances in programming language theory and mathematics in order to construct a simple and novel view of computation.
%Our paradigm is formulated around the synthesis of definitions and queries through the well-known Curry-Howard correspondence.
%%It is our opinion that this archetype expresses the nature of computation well due to the orthogonal, yet reciprocal nature of these two concepts.
%It is our opinion that this archetype can be usefully employed to express computer programs due to the orthogonal, yet reciprocal nature of these two operations.

%%We show that concepts from catagory theory can be related to . 
%The model that we propose is described as a formal language designed with explicit reference to generic programming, non-determinism, sequential processes, constraint modelling and proofs as programs.

%%The language is based on 
%%propose a new formal language as a foundation for building new programming languages.
%%We speculate that the concepts of definitions and queries as a pragmatic form two central and orthogonal requirements of any formal language.
%%This paper is intended as exploratory research.
%%\emph{(TODO: Exploratory research...?)}
\end{abstract}

\textbf{Keywords:} Formal language, Computational theory, Self-refrencing definition	

\section{Introduction}

\subsection{The nature of computation}
\subsubsection{Universality}
\begin{itemize}
\item \emph { Universal Turing  Machine }\\
\item \emph { The Lambda Calculus }\\
\item \emph { The Halting problem }\\
\item \emph { Impredicativity: http://en.wikipedia.org/wiki/Impredicativity }\\
\item \emph { An axoimatic foundation of choice: Zermelo–Fraenkel set theory  - Note the foundational crisis of mathematics.}
\end{itemize}

\subsection{Pragmatics}
\subsubsection{Computer languages}

While proof of universality for a combination of operations allow us to identify the languages in which all known computation is \emph{possible} it is not the only metric by which we can measure the pragmatic importance of such a system.
These metrics are necesarily subjective requirements that must be specified rather than derived.

\begin{itemize}
\item Expressivenes: The level of self-reference available to a programming language.
\end{itemize}

\section{Introduction}
In the initial sections of this paper we will ignore certain existing. In the final sections a comparison with the existing theories is provided.

\section{Concepts}

\begin{itemize}
\item Curry-Howard correspondence: Once one starts investigating, similar correspondences can be found in a very wide range of programming concepts.
\end{itemize}

\subsection{Defintion and queries}
\subsubsection{Existence as the fundamental assertion}
We propose that the most fundamental definition that can be understood is an assertion of existence. 
By simply listing any unreserved token in our language, we will claim that it exists in some \emph{context}.
Where this idea is used mathematical notation or requires explicit elucidation in commentary it will be indicated using the $\exists$ symbol.
For example, an assertion that the user-defined token $a$ exists will be written $\exists a$ in the documentation.
Inversely, claiming that $a$ does not exist can be indicated as $\nexists a$.
Mathematically the assertion that multiple tokens exists can be indicated in a single expression in this fashion: $\exists a \land \exists b \land \exists c$.
In our basic language, simply listing these elements as \texttt{a b c} is equivalent.

%\begin{itemize}
%\item \emph { $a \neq \emptyset$ means $a$ exists. }
%\item \emph { $\exists a . a \in [a\ b\ c] \equiv [a\ b\ c] \cap [a] \neq \emptyset$ means ``$a$ exists in the set $[a\ b\ c]$''. }
%\end{itemize}
%\emph { We use the symbol $\emptyset$ interchangeably with the more general symbol $\bot$ commonly used with partial ordering and denotational semantics. }\\

%This concept is easily understood in a pragmatic manner. 
%For example, take a simple file that contains a single data element \{\texttt{5}\}. 
%Consider what this means to the casual observer.
%By assuming that all external references are hidden from the observer, we can only assume that the data element \texttt{5} exists.
%We will say that \emph{\texttt{5} exists within the context of the file}. 
%Similarly any \emph{data} is simply an assertion of the existence of said data within the \emph{context} in which it is defined.

\subsubsection{Source, state and denotational representations}
The difference between mathematical formalism and computational formalism is that the mechanical procedure of converting data from one representation to another becomes relevant.
In programming language design denotational semantics are used for the first whereas operational semantics is used for the second.

Our approach in this paper will be a little different. We will use an \emph{annotated} syntax to indicate various representations, giving the reader a concise idea of the domain that the data inhabits.

Let us look again at our fundamental assertion of existence. There are three environments in which such an assertion must be considered.
\begin{itemize}
\item The first is the \emph{unoccupied} representation $\textlquill$a b c$\textrquill$
\item An \emph{occupied} representation $\rightarrow$ <<a b c>>.
\item ... 
\end{itemize}
We have already given a 

Internal annotation...

More precisely:



\subsubsection{Querying existence}
Following on the previous definition we are inclined to ask three questions.

\begin{itemize}
\item \emph{Which tokens exist?} The only sensible manner of answering this question is by reproducing the definitions already given to us. 
This will be written using the external 
\item \emph{Which tokens do not exist?}
\item \emph{Which tokens do not exist?}
\end{itemize}

A notation $Q\llbracket a \rightarrow b \rrbracket = a \mapsto a:b$ will be used

$Q\llbracket \ \rrbracket \mapsto a$


\subsubsection{Metadata}
While existence is a 

\subsubsection{Inference as context}



\subsubsection{Inference as context}
By listing symbols outside of any context we will simply claim that they exist.\\




We propose that the most fundamental definition is an assertion of existence. By listing symbols outside of any context we will simply claim that they exist.\\
For example:
\begin{itemize}
\item $a$: means ``$a$ exists''.
\item $a\ b\ c$: means ``$a$, $b$ and $c$ exist''.
\end{itemize}

In addition to simply defining symbols we would like to 
Recall that the Curry-Howards [citation needed] correspondence states that proofs are programs refering to the suprisingly pervasive connection that mathematics has to logic.
If we claim that the basic building block of mathematics are functions then we could imagine that the basic building block of logic is induction.
Indeed, curry-howard [citation needed] showed that functions correspond to logical implication. 
It is tempting to cast these two dual (?) constructs into the roles of queries and definitions respectively.\\

In logic the operator $\rightarrow$ can be defined as follows:
\begin{itemize}
\item $a \rightarrow b$: means ``the truth of $a$ implies the truth of $b$''. 
Following our previous conjecture of existence being the fundamental assertion we will refine our meaning slightly to ``the existence of $a$ implies the existence of $b$'' where truth values 
can be converted to an assertion of existence simply by converting to the corresponding set operation usually through set operations such as intersection and union in accordance with the curry-howard correspondence.
\end{itemize}

Functions in mathematics have a long history and many different representations including the lambda calculus.
In order to clearly establish the connection to sets and logical implication, we will define functions as sets of mappings such that elements on the left hand side of the 
$\mapsto$ operator will be replaced by elements on the right-hand side when it is bound. (Note: we will discuss binding later... this will be slightly different from the lambda calculus since we will be using intersect-binding in sets)

\begin{itemize}
\item $a \mapsto b$: means ``$a$ will be replaced by $b$''. 
\item $a \mapsto [b\ c\ d]$: means ``$a$ will be replaced by the set $[b\ c\ d]$''. 
\end{itemize}

Let us then define an isomorphism $Q$ that converts a definition into a query as follows:\\

$Q\llbracket a \rightarrow b \rrbracket = a \mapsto b$

and consequently

$Q^{-1}\llbracket a \mapsto b\rrbracket = a \rightarrow b$


Category theory similarly makes use of `Arrows' and monads... (todo)


However we also wish to model induction as the basic.

\begin{itemize}
\item $a \rightarrow b$: means ``the existence of $a$ implies the existence of $b$''.
\item $a \rightarrow [b\ c\ d]$: means ``the existence of $a$ implies the existence of $b$, $c$ and $d$''.
\end{itemize}

Equivalence
$a = b$

Context
$a = [x y]$ means "$x$ and $y$ exists within the context of $a$. (Or perhaps even, if $a$ exists then $x$ and $y$ exists)"

Contextual equivalence
$a = [(x = i) (y = j)]$

\subsubsection{ A comparison with structural (operational) semantics }

%---
%$a$

%--------------------
%$a \land b \land c$

%-------------   
%$a \mapsto x$

%which becomes 

%---------------
%a \land b \land a = b


%$a$\\
%-----------\\
%$x \land y \land i \land j x \mapsto i \land  y \mapsto j$



%TODO: Using queries
%a
%a.b = d
%a.c = e



\section*{Assertions as type systems}
"Labeling"

\section*{Assertions as containers}

\section*{Recursive types (meta types)}

\section*{Infinite types}

\section*{Infinite types as functions}
Note: discuss fixed points here...

\section*{Terminating type systems}
It is a well known result that no program can be constructed that can prove whether any other program will terminate. [citation needed: turing]
We say that the problem of determining termination is undecidable in general.
It is our contention that whenever a problem that is seemingly intractable crops up it is often the case that we are asking the wrong question.
Instead a more useful proposition would be to ask whether we can construct programs in a turing-complete language and yet be certain that it will terminate.
At first this may appear to be the same problem merely stated in a different way, but consider the inverse implication that if we could not write programs that we would be certain would terminate.
Clearly we \emph{can} write programs that we are certain would terminate or would not terminate. 
There are many trivial examples of both that can be considered.
The problem then, is reduced to merely guaranteeing that we \emph{do} write programs with pre-determined termination or non-termination characteristics.
Indeed nontermination is desireable in some subprocesses where a parent process can force termination. 
An interpreter for another language for one thing can be viewed as a non-deterministically terminating type. 
If the child process halts then the interpreter will halt. 
However, the user may also choose to stop the interpreter. In which case the child process will also be stopped.
Hence there are two views of termination: One which is internally non-deterministic and one which is externally non-deterministic.
During the construction of programs, such choices form the guiding design principles in constructing correctly.
Essentially it is a engineering problem of `where' to allow non-determinism. We could choose to provide an interpreter that only execute terminating programs with limited computational power or allow non-terminating 
programs with full computational power.

To understand this concept, let us state the halting-problem in different words:
There exists infinitely many programs of infinite complexity. In order to prove for any of these programs that they will terminate we require a infinitely complex prover program.
However, for any program with finite complexity there exists a proof program of finite complexity that can show whether this program terminates depending on some 
In fact these are the only programs in which we are interested in because these are the only programs that we can realistically write. 
A problem is that additional complexity can be injected into programs through user input which implies that additional complexity must be injected into the prover program in order deal with it.

(Note: Writing programs that are not provable given all inputs is itself a undecidable problem?)
Hence, if this idea could be implemented in practice we will claim that our programs are either terminating or non-terminating \emph{by construction}.
In order to guarantee construction we introduce the concept of terminating and non-terminating type systems.
The previous section the concept of infinite types were introduced and a comparison to functions was given.
Let us substitute the term `infinite' with the term `non-terminating'. A terminating type can be constructed from a non-terminating type by placing some upper bound on the infinite type.
In essence, the burden of proof is moved to the user.
Now only the programs that we can construct are those we can prove, but this is exactly what we want rendering all other programs irrelevant.
It is proposed that there are programs we that cannot write and hence cannot prove termination either.

TODO: Attempt a formal proof using turing machines or similar...? Prove that any program generated by a turing machine can be proven by the same turing machine will terminate if no outside inputs are given.


Furthermore, termination is not always a desireable feature

\section*{Compilation}
It is the responsibility of the \emph{compiler} to convert the definition graph into a state graph in a way such that all side effects that can occur, may occur. The compiler

Using our language framework it is possible that the compiler may be engineered as a \emph{reflective} builder in the sense that a user extends the compiler rather than merely using the compiler as a tool.
This would allow the user to create DSL's and even specify their own "compile-time" optimizations specific to the domain and type system engineered by them.
However, we anticipate that if this is attempted some form of library system would needed to include pre-compiled code as a library rather than.

The level of static compilation is dependent on what a user can input into the system. JIT compilation should be availalbe for scripting purposes, however some mechanism is needed to hide static data 
from the graph while maintaining those parts that may be exposed to user input.
This is both a performance and a security concern.



\section*{Conclusions}

The language framework unifies 4 major programming paradigms: 
\begin{itemize}
\item Imperative programming through side-effects and a state transition system
\item Functional and meta programming which converts definitions to queries and implements queries to construct new definitions.
\item A proof assitant. Definitions and queries are coincide through the Curry-Howard isomorphism. (TODO: But how is correctness checked? By expansion and the type system possibly).
\item Logic / Constraint-based programming. Queries may be run through a constraint solver. (also used to compile the state transition system and finally the AST of a static program. I.e. a self-compiler)
\end{itemize}

Even though language designers frowns on use of the word "high-level" to describe programming languages we would never-the-less like to use it here. 
Each paradigm in the above list represents a distinct tier in language capability in the sense that every item transfers responsibility from the human programmer to the compiler and/or libraries. NOTE: 

SIDE NOTE (not for paper): A speculative question then is, what is the next tier? Since we've defined "high-level" as handing responsibility to the compiler the only logical conclusion is that the next level would involve
the compiler inferring constraints by itself which requires what we characterize as "creativity" or perhaps even "autonomy". Practically, this probably implies some form of fuzzy constraints \& adapting to changing requirements or the behaviour of 3rd party agents.
Note that constraints in this context corresponds to "specifications" or "requirements".

\section*{Future work}

\bibliographystyle{plain}
\bibliography{delayedcase}

\end{document}

